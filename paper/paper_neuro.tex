\documentclass[10pt,journal,compsoc]{IEEEtran}
\usepackage{xcolor}
\usepackage{hyperref}
\usepackage{ulem}
\usepackage{graphicx}
\usepackage{rotating}
\usepackage{setspace}
\usepackage{longtable}
\usepackage[binary-units=true]{siunitx}
\usepackage{multirow}
\usepackage{subcaption}
\usepackage{algorithm}
\usepackage{algorithmicx}
\usepackage{algpseudocode}
\usepackage{amsmath}

\algnewcommand\algorithmicforeach{\textbf{for each}}
\algdef{S}[FOR]{ForEach}[1]{\algorithmicforeach\ #1\ \algorithmicdo}
\algblock{Input}{EndInput}
\algnotext{EndInput}
\algblock{Output}{EndOutput}
\algnotext{EndOutput}
\newcommand{\Desc}[2]{\State \makebox[2em][l]{#1}#2}
\newcommand{\todo}[1]{\marginpar{\parbox{18mm}{\flushleft\tiny\color{red}\textbf{TODO}: #1}}}

% benchmark on beluga, prefreesurfer with RUIS
% use more than 64 subjects for prefreesurfer --> few 100 subjects
% think more about how to benchmark I/O
% try distributed jobs with Sea

\begin{document}
\title{Automated neuroimaging data management with the Sea filesystem}
% remove neuroimaging and replace with scientific computing  add user-space, switch HPC to computing clusters

\author{Val\'erie Hayot-Sasson and Tristan Glatard}

% The paper headers
\markboth{Journal of \LaTeX\ Class Files,~Vol.~14, No.~8, August~2015}%
{Hayot-Sasson\MakeLowercase{\textit{et al.}}: Bare Demo of IEEEtran.cls for Computer Society Journals}
\IEEEtitleabstractindextext{%
\begin{abstract}
    Neuroimaging open-data initiatives have led to the availability of large scientific datasets. These voluminous
    datasets provide researchers with new insights into the human brain across diverse populations. Difficulties with
    managing such large data has partially hindered the advancement of scientific studies analyzing these datasets. Many
    software tools and strategies have emerged to facilitate the processing. Many open datasets have been stored on cloud storage
    services, such as AWS, enabling rapid transfer of the data at a cost. Software tools such as Datalad has facilitated both the
    sharing and versioning of data. Despite these advancements in-analysis data-management remains limited in neuroimaging workflows.

    Neuroimaging workflows, particularly preprocessing ones, may lead to a magnification in output sizes due to <add reason>. They also produce
    significant amounts of intermediary data due to being composed of a multitude of steps. 
\end{abstract}
}


% make the title area
\maketitle


\IEEEdisplaynontitleabstractindextext
\IEEEpeerreviewmaketitle



\IEEEraisesectionheading{\section{Introduction}\label{sec:introduction}}
\IEEEPARstart{}{}

Neuroimaging datasets continue to grow in size resulting in new challenges related
to data management. The current largest neuroimaging datasets, such as the Human Connectome Project (HCP)~\ref{HCP}
and the UK Biobank~\ref{ukbiobank} reach up to Petabytes of data. Big Data in neuroimaging can be present in two formats: 1)
very large files, such as those found in the BigBrain~\ref{bigbrain}, and 2) large datasets made up of very small files,
such as those typically found in fMRI datasets. 

When processing, large datasets can result in longer processing times even when compute resources are ample.
This is a direct result of the underlying storage used by the applications to read and write data. Since large
datasets require a significant amount of available storage space, more often than not, slower but larger storage is
selected, resulting in longer data transfer times during processing. With the Sea filesystem, we aim to
leverage all available storage in order to reduce overall data transfer time during pipeline execution.

Many researchers rely on one of two systems to meet their storage and computing needs: 1) High Performance 
Computing (HPC) clusters and 2) the cloud. Whereas the cloud simplifies data sharing and gives researchers
access to a wide variety of infrastructures, HPC clusters are a cost-effective solution to accessing a wide
array of resources for researchers. In this paper, we will focus on HPC processing.

HPC systems rely on scalable network-based parallel filesystems (e.g. Lustre) for storage. While such file
systems offer excellent performance, they are shared between all the users on the cluster. Meaning that users
with data-intensive can effectively deteriorate the performance of the shared file system for all users on
the cluster. Solutions for improving shared file system performance include throttling the data intensive workloads
or recommending the use of Burst Buffers~\cite{bb} (e.g. reserving a compute node for storage or leveraging
local compute storage during processing). The latter, however, requiring that the user manages their data to
and from the Burst Buffer.

Leveraging local storage to improve data intensive workload performance has long since existed in Big Data
frameworks such as Apache Spark~\cite{spark} and Dask~\cite{dask}. While these frameworks have been used to
process neuroimaging data~\cite{manypapers}, they remain seldom used as their require rewriting existing
standard neuroimaging tools (e.g FSL, AFNI, SPM) for the framework. Although it is possible to use the
tools within the Big Data frameworks, optimizations like in-memory computing would not be leveraged due to 
the fact that neuroimaging tools are meant to be used as command-line tools and do not provide interfaces that 
enable the data to be transferred in-memory.

Frameworks used in neuroimaging, such as NiPype~\cite{nipype} and Joblib~\cite{joblib} instead focus on reducing compute times of workloads.
This is because even with large datasets, neuroimaging data processing remains split between compute and data
intensive components. Although these frameworks do not prohibit the use of intelligent data management, it is not directly integrated
into the workflow. In order to give neuroimaging applications data management capabilities, the applications must interact with a
file system that can do so.

In order for a file system to be usable by the average researcher on an HPC system, they must be loadable without administrative
privileges. Furthermore, as the applications are typically made to interact with POSIX-based file systems, the new file system must
also be compliant to the format. One method to ensure that these conditions are met is by using the \texttt{LD\_PRELOAD} trick. This trick
enables the interception of library calls by redefining them. 

In this paper, we present Sea, a file system designed to automate data management of neuroimaging pipelines running on HPC systems.





\section{Related Work}

Put in intro.
summary of computing landscape for neuroimaging.
HCP, CBRAIN, nipype, joblib. sea can be used in all these contexts.


\section{Materials and Methods}
\subsection{Datasets}
Functional magnetic resonance imaging datasets can vary greatly in total number
of images and number of volumes within each image. To adequately capture the 
diversity of datasets and when Sea could be pertinent, we selected three datasets of varying
sizes: 1) OpenNeuro's ds001545 dataset~\cite{ds001545}, 2) the PREVENT-AD  dataset~\cite{preventad}, and
3) the HCP dataset~\cite{hcp}. The ds001545 dataset is a total of 45.94~$GB$ (1778 files) and consists of data collected
from 30 subjects in a single session. The PREVENT-AD dataset is a 255.0~$GB$ dataset (53061 files) consisting of data collected
from 308 subjects. The HCP dataset is the largest of the three datasets at 85.4~$TB$ (x files) consisting of data collected from
1113 subjects.

%just for example run
%dataset size, resolution, resolution of voxels
%matrix size pixdim1X2
%number of slices pixdim
%number of volumes pixdim4
%inplane resolution pixsize1x2
%slice thickness pizsize3 in mm
%repetition time pixsize4 in s

%HCP resting state + x stats

%# files written and output size, duration and memory consumption.
\subsection{fMRI Preprocessing Pipelines}

Similarly to datasets, preprocessing pipelines can also vary in duration as a result of methodological differences.
Thus, we preprocessed each dataset using four standard preprocessing pipelines: FSL~\cite{fsl}, AFNI~\cite{AFNI},
SPM~\cite{SPM}, fMRIPrep~\cite{fmriprep}. Table~\ref{tb1} provides an reference of the differences in computation and data-intensivity
of the different pipelines on a single subject of the ds001545 dataset.

Each tool was set to only run the preprocessing pipeline. Scripts used to execute each pipeline can be found at~\ref{github.com}.
For the FSL preprocessing pipeline, the default settings were maintained with the exception of slice-timing correction (interleaved),
intensity normalization and non-linear registration to standard space.

%% Add SPM here

For AFNI preprocessing, the 

%%{provide subject #/session#} of the ds001545 dataset.



\subsection{Controls}
\subsection{Infrastructure}

\section{Results}


\section{Discussion}
talk about testing


Flushing - rely on modification time
\section{Conclusion}
%


\ifCLASSOPTIONcompsoc
  % The Computer Society usually uses the plural form
  \section*{Acknowledgments}
\else
  % regular IEEE prefers the singular form
  \section*{Acknowledgment}
\fi

% Can use something like this to put references on a page
% by themselves when using endfloat and the captionsoff option.
\ifCLASSOPTIONcaptionsoff
  \newpage
\fi



% biography section

\begin{IEEEbiography}{Val\'erie Hayot-Sasson}
\end{IEEEbiography}
\begin{IEEEbiography}{Tristan Glatard}
\end{IEEEbiography}
\end{document}


