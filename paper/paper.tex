\documentclass[10pt,journal,compsoc]{IEEEtran}
\usepackage{xcolor}
\newcommand{\todo}[1]{\marginpar{\parbox{18mm}{\flushleft\tiny\color{red}\textbf{TODO}: #1}}}
\begin{document}
\title{Sea: A hierarchical filesystem for Big-Data Scientific Computing}
% remove neuroimaging and replace with scientific computing  add user-space, switch HPC to computing clusters

\author{Val\'erie Hayot-Sasson and Tristan Glatard}

% The paper headers
\markboth{Journal of \LaTeX\ Class Files,~Vol.~14, No.~8, August~2015}%
{Hayot-Sasson\MakeLowercase{\textit{et al.}}: Bare Demo of IEEEtran.cls for Computer Society Journals}
\IEEEtitleabstractindextext{%
\begin{abstract}
\end{abstract}
}


% make the title area
\maketitle


\IEEEdisplaynontitleabstractindextext
\IEEEpeerreviewmaketitle



\IEEEraisesectionheading{\section{Introduction}\label{sec:introduction}}
\IEEEPARstart{E}{fficient} data-management strategies have become imperative in 
mitigating the costs of processing large datasets. These strategies come in the form
of data locality, which ensures that computation is performed where the data resides
rather than relocating the data, and in-memory computing, which ensures that the data is
maintained in RAM, whenever possible. Big Data frameworks, such as the Hadoop MapReduce, Spark and Dask,
have evolved to support either or both of these strategies.
However, while many domains have adopted the use of Big Data framework and filesystems to ensure effective data
management within their applications, many scientific applications are developed without such strategies in mind.
% add speedup results

Scientific applications may omit leveraging Big Data frameworks, despite having Big Data workloads,
for many reasons. One such reason may be the data format. BigData frameworks were originally developed
for text-processing. Scientific applications, in contrast, may rely on image processing for their
analyses. The fundamental differences between these two data formats made leveraging BigData frameworks for
scientific image processing non-trivial. While recent efforts have been made to support data-intensive
image processing workloads, there remains some work to be done.

Scientific applications are usually composed of numerous well-established command-line tools. These tools
may have been developed at a time when the processing would have been compute intensive, thus not requiring any
data management strategies, yet remain scientifically relevant. While newer applications
that leverage such tools may be written using Big Data frameworks, such frameworks are not designed to transfer
data in-memory to command-line applications, and thus would require some instrumentation.

While various infrastructures are available to researchers, including local workstations, cloud and 
High Performance Computing (HPC) clusters, researchers commonly rely on HPC clusters due to their low cost and available
resources. Such clusters typically rely on a network-based parallel file system, such as Lustre, Ceph and GlusterFS.
Since data storage and compute are maintained as separate entities on HPC clusters, processing BigData workflows on them
may be particularly costly, particularly when no BigData framework is applied. Furthermore, while it is possible to execute
workloads using BigData frameworks on HPC, these frameworks are typically incompatible with the native resource managers (e.g. SLURM, PBS),
and therefore, and overlay cluster must be used.

In this paper we introduce Sea, a user-space hierarchical file system for scientific workloads. The aim of Sea
is to bring BigData performance to scientific workloads through the incorporation of data locality and
in-memory computing. Unlike Big Data frameworks, Sea can be utilized alongside existing workloads without
the need to entirely reinstrument the existing workload. Furthermore, as it executes entirely in userspace,
it does not require elevated privileges to be installed and executed.

%multiple disk support
% pipeline agnostic and pipeline access patterns
% don't leverage workflows
% leverage existing filesystems
% replace storage devices with storage location.
% use applications
% memory management paragraph: PAGE CACHE
% shared filesystems are slow and metadata management can be slow




% add need for userspace / intelligent pipeline flushing
% there is no one-size fits all with data intensive applications
% due to io patterns, so we decided with userspace.


% add related work section
\section{Related Work}
\subsection{Lustre}
Shared parallel filesystems
% change lustre to shared filesystem
\subsubsection{Other Luster optimizations}
\subsection{Big Data Frameworks and File systems}
HDFS Alluxio Hadoop Spark Dask
\subsubsection{What are they/how do they work}
\subsubsection{Why are the suboptimal for neuroimaging/on HPC}
\subsection{The linux page cache}
\subsubsection{What is it/how does it work}
\subsubsection{Why do we still need Sea}
\subsection{File System implementations}
\subsubsection{Kernel-space file systems}
\subsubsection{Burst Buffers}
\subsubsection{Unionfs}
\subsubsection{FUSE}
\subsubsection{System call interception with ptrace}
\subsubsection{LD\_PRELOAD}
xtreemfs/xtreemos

\section{Materials and Methods}

\subsection{Sea design and implementation}
\subsubsection{Overview}

Sea is a Linux-compatible C++ library that redirects files accessed from a user-specified mount point to the
appropriate storage devices. In order to effectively redirect all data transfers, it leverages the LIBC\_PRELOAD functionality which
enables the interception of libc calls. Within an intercepted call the file will be written to or read
from the best available storage device. At minimum, a user must specify at least two storage devices, a fast
temporary one and a slower long-term storage one. This could be RAM and SSD if working on a single node, or a compute-local SSD
and a shared parallel file system, in the case of an HPC cluster. Ideally, a user will provide a multitude of short-term storage
devices to improve Sea's efficacy. Furthermore, to maximize usage of the fastest available
storage devices, Sea will allow the user to outline which files can be removed from short-term storage in addition
to which files need to be materialized onto long-term storage.

In this subsection, we will discuss and outline the various design and implementation details made. Furthermore,
we will go into detail into the functionality of Sea and how it should and should not be used.

\subsubsection{Requirements and Assumptions}

Sea is primarily designed to work with workloads that generate copious amount of intermediate data.
Since data must be read from the slower storage device and final output data most likely would have to
be written to the same device, the use of Sea with workloads that do not produce intermediate data would
bring no speedup and may even introduce some overheads. 

Sea provides two main modes that are dictated by how the user specifies flushing and eviction: In-memory computing and flush-all.
In-memory computing is achieved when intermediate need not be materialized to long-term storage. In these scenarios, Sea can fully
leverage local (short-term) storage to maximize performance. This can be coupled with eviction to limit the amount of
data that will be written to slower, but larger local storage devices. With Sea in memory, we can expect performance comparable to
that of using BigData frameworks.

Occasionally, it is desired to materialize even intermediate data to long-term storage devices, for potential future use. The
performance gain from using Sea in these scenarios, however, is largely bounded by data size. In the instances where the workloads is more compute intensive,
we can gain speedup by masking all or part of the I/O through long-term storage by executing the flush during a compute task. In data-intensive scenarios, the 
opposite can be seen: compute time can be masked by I/O (add figure + model). Despite the performance gain being limited in
such a scenario, a performance gain can still be observed through the use of Sea.

One important assumption that must be made is that the amount of data produced by the workload far exceeds the amount of page cache space available and utilized
by the different filesystems. In the case where all workload data can fit into page cache memory, the benefit of Sea may be
greatly diminished if not negligible.

Depending on the version of libc used, function calls may differ. Sea is currently compatible with libc version X-X.


\subsubsection{Benefits}
Sea alleviates the user of the need to implement the logic to redirect data to the appropriate
storage locations. Furthermore, it implements logic to ensure that files are written to the best possible
location at any given time. It is both pipeline and infrastructure agnostic. 

As it executes in user-space, root permissions are not required, enabling users to use Sea on most linux-based systems
available to them. This differs from Big Data filesystems which may require elevated permissions to install.
The overhead of intercepting libc calls
is minimal, and negligible compared to system call interception and file systems such as FUSE.

Unlike BigData frameworks, Sea does not require reinstrumentation of the existing pipelines, allowing
users to gain an instant performance boost.

\subsubsection{libc interception}

libc interception is achieved by writing wrappers to existing libc functions. Most importantly, every
libc function accepting a file path needs to be wrapped. The wrappers' job is to take any input filepath
that is located within the user-provided Sea mountpoint and convert it a filepath pointing to the best available storage device.

Sea currently does not support the partitioning of files across multiple devices. Since it cannot predict the size of the outputs to ensure the
existence of sufficient space on storage devices, the user must provide within the Sea configuration file the maximum file size produced by the workflow.
Together with the specified amount of parallel processes, Sea calculates the minimum space required on a storage device to write the file to it \todo{should include an equation here}.
Sea will then go through the hierarchy of available storage devices an select the fastest storage device with sufficient available space.

While this process may result in not utilizing all available space in fast storage devices, it will still allow users to gain a performance speedup on the pipeline given that the
number of threads multiplied by the file size does not exceed storage space.


\subsubsection{Flushing and eviction}
\todo{add diagrams here?}
There are four different available usage scenarios: flush, evict, flush and evict and do nothing.
A Sea flush is synonymous to a file copy in that files that are stored within Sea but have not yet been materialized
to the long-term storage device will be copied to long-term storage. Flushing generally required for all intermediate data
that are required for future analysis.

Eviction, on the other hand, depicts a remove operation. Any file that is evicted from Sea will be removed from any of the storage mount
points located in Sea. 

\subsubsection{Configuration File}
\subsubsection{Program execution}
\subsubsection{Limitations}
behaviour unspecified in mixed static/dynamic workloads

\subsection{Experiments}
\subsubsection{Infrastructure}
\subsubsection{Pipelines}
Big Brain incrementation
fmri processing?

\section{Results}


\section{Conclusion}
The conclusion goes here.





% if have a single appendix:
%\appendix[Proof of the Zonklar Equations]
% or
%\appendix  % for no appendix heading
% do not use \section anymore after \appendix, only \section*
% is possibly needed

% use appendices with more than one appendix
% then use \section to start each appendix
% you must declare a \section before using any
% \subsection or using \label (\appendices by itself
% starts a section numbered zero.)
%


\appendices
% you can choose not to have a title for an appendix
% if you want by leaving the argument blank
\section{}
Appendix two text goes here.


% use section* for acknowledgment
\ifCLASSOPTIONcompsoc
  % The Computer Society usually uses the plural form
  \section*{Acknowledgments}
\else
  % regular IEEE prefers the singular form
  \section*{Acknowledgment}
\fi

% Can use something like this to put references on a page
% by themselves when using endfloat and the captionsoff option.
\ifCLASSOPTIONcaptionsoff
  \newpage
\fi



% biography section

\begin{IEEEbiography}{Val\'erie Hayot-Sasson}
\end{IEEEbiography}
\begin{IEEEbiography}{Tristan Glatard}
\end{IEEEbiography}
\end{document}


