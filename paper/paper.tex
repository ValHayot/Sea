\documentclass[10pt,journal,compsoc]{IEEEtran}

\begin{document}
\title{Sea: A hierarchical filesystem for processing Big Neuroimaging Data on HPC}

\author{Val\'erie Hayot-Sasson and Tristan Glatard}

% The paper headers
\markboth{Journal of \LaTeX\ Class Files,~Vol.~14, No.~8, August~2015}%
{Hayot-Sasson\MakeLowercase{\textit{et al.}}: Bare Demo of IEEEtran.cls for Computer Society Journals}
\IEEEtitleabstractindextext{%
\begin{abstract}
\end{abstract}
}


% make the title area
\maketitle


\IEEEdisplaynontitleabstractindextext
\IEEEpeerreviewmaketitle



\IEEEraisesectionheading{\section{Introduction}\label{sec:introduction}}
\IEEEPARstart{N}{euroimaging} datasets have significantly grown in size in the past decade,
resulting in Petabytes of data. As a result, processing pipelines have now shifted from compute to 
data intensive. Some efforts have been made to reduce the data transfer costs in neuroimaging by
introducing new methods to generate pipelines, such as through the adoption of Big Data frameworks
like Apache Spark \cite{Spark} for the pipeline definition \cite{thunder} \cite{other papers}. 
However, such pipelines would have to be entirely rewritten for the framework of choice and Big Data
frameworks like Spark and Dask were found to be difficult to use in describing neuroimaging pipelines \cite{mehta}.
Furthermore, the neuroimaging community relies heavily on the use of well-established command-line tools \cite{fsl, afni, freesurfer}, resulting in
data transfers being inevitable, even with a Big Data framework. 

High Performance Computing (HPC) clusters are often the infrastructure of choice for researchers looking to processing their data.
While there are many benefits to using such clusters, such as large amounts of resources available at little to no cost to
the researchers, the resources are shared amongst users concurrently. 

Goal - leverage local storage to speed up computation/Sea

\subsection{Lustre}
\subsubsection{Burst Buffers}
\subsubsection{Other Luster optimizations}
\subsection{Big Data Frameworks and File systems}
HDFS Alluxio Hadoop Spark Dask
\subsubsection{What are they/how do they work}
\subsubsection{Why are the suboptimal for neuroimaging/on HPC}
\subsection{The linux page cache}
\subsubsection{What is it/how does it work}
\subsubsection{Why do we still need Sea}
\subsection{File System implementations}
\subsubsection{Kernel-space file systems}
\subsubsection{FUSE}
\subsubsection{System call interception with ptrace}
\subsubsection{LD\_PRELOAD}

\section{Materials and Methods}

\subsection{Sea design and implementation}
\subsubsection{Requirements and Assumptions}
\subsubsection{Benefits}
User does not need to create logic to determine what file systems to write to / implement flushing

\subsubsection{libc interception}
\subsubsection{Flushing}
\subsubsection{Configuration File}
\subsubsection{Program execution}
\subsubsection{Limitations}

\subsection{Experiments}
\subsubsection{Infrastructure}
\subsubsection{Pipelines}
Big Brain incrementation
fmri processing?

\section{Results}


\section{Conclusion}
The conclusion goes here.





% if have a single appendix:
%\appendix[Proof of the Zonklar Equations]
% or
%\appendix  % for no appendix heading
% do not use \section anymore after \appendix, only \section*
% is possibly needed

% use appendices with more than one appendix
% then use \section to start each appendix
% you must declare a \section before using any
% \subsection or using \label (\appendices by itself
% starts a section numbered zero.)
%


\appendices
% you can choose not to have a title for an appendix
% if you want by leaving the argument blank
\section{}
Appendix two text goes here.


% use section* for acknowledgment
\ifCLASSOPTIONcompsoc
  % The Computer Society usually uses the plural form
  \section*{Acknowledgments}
\else
  % regular IEEE prefers the singular form
  \section*{Acknowledgment}
\fi

% Can use something like this to put references on a page
% by themselves when using endfloat and the captionsoff option.
\ifCLASSOPTIONcaptionsoff
  \newpage
\fi



% biography section

\begin{IEEEbiography}{Val\'erie Hayot-Sasson}
\end{IEEEbiography}
\begin{IEEEbiography}{Tristan Glatard}
\end{IEEEbiography}
\end{document}


